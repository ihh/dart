\documentclass{article}
\usepackage{psfig}
\begin{document}

\newcommand\Telegraph{{\tt telegraph}}

\title{Telegraph: Machine-Readable Probabilistic Grammars}
\author{Ian Holmes}
\maketitle

\abstract{{\bf Motivation.}
\\
{\bf Methods.}
\\
{\bf Results.}
\\
{\bf Availability.}
\\
{\bf Contact.}
\\
}

% Introduction.
\section{Introduction}

Telegraph is a functional language based on C and {\tt yacc} (and to a lesser
extent C++ / IDL) and is best understood by direct comparison to those
languages.

\subsection{C-like types}

\begin{verbatim}
	bool
	int
	double
	unsigned int
	unsigned double
	enum
	struct
\end{verbatim}

These are the same as in C, except for {\tt unsigned double} (whose meaning
should be obvious; it is useful for probabilities) and {\tt bool} (boolean,
as in C++). Also, {\tt enum}'s can't have explicit integer assignments, as they
can in C, i.e. you can write

\begin{verbatim}
		enum ascii { A, B };   /* legal */
\end{verbatim}

but not

\begin{verbatim}
		enum ascii { A = 65, B = 66 };   /* illegal */
\end{verbatim}

Operators that work on C types generally work on the corresponding
Telegraph types, except for assignment operators like {\tt +=}
which change the value of a variable (since Telegraph is functional,
variables can't change their values).

An important exception is the equality comparison operator {\tt ==} which only
works on {\tt enum}, {\tt int} and {\tt bool} in Telegraph, but not {\tt double}. In other
words, {\tt ==} works on discrete types but not continuous types. (In
practise, using {\tt ==} on {\tt double}'s is risky even in C, due to rounding
errors. I've had code break due to this before.)

The distinction between discrete and continuous types is an important one
in Telegraph \& will be returned to. Essentially, discrete types have a
finite range of values and so can be used to index bounded lookup tables.

\subsection{Type constructors}

\begin{verbatim}
	sequence<X>
	table<K,V>
	int (A..B)
	G parse (S1 [,S2 [,S3...]])
\end{verbatim}

The first two of these are used to build complex types from simpler types,
like generic programming in C++. The second two build types from
expressions \& thus are more functional in nature.

\begin{verbatim}
	sequence<X>
\end{verbatim}

This is similar to IDL's {\tt sequence<X>} (and C++'s {\tt vector<X>} in the
standard template library).

\begin{verbatim}
	table<K,V>
\end{verbatim}

{\tt table<K,V>} is a map from a key type ({\tt K}) to a value type ({\tt V}), similar to
C++'s {\tt map<K,V>}. The difference is that in C++, {\tt K} can be any model of
EqualityComparable (i.e. any type supporting the {\tt ==} operator) whereas
in Telegraph it must be a discrete type. (Note that only discrete types
support {\tt ==} in Telegraph, so this is consistent.)

Another difference is that in C++, {\tt map}'s are initially empty (and
key-value pairs are added later), whereas a Telegraph table defines a
value {\tt V} for every {\tt K}. Thus a {\tt table} is actually more like a C array, and
I expect most compilers will represent {\tt table}'s as arrays.

\begin{verbatim}
	int (A..B)
\end{verbatim}

This type represents an integer in the range {\tt A..B}, where {\tt A} and {\tt B} are
expressions. For example, if {\tt PROT} is a protein sequence, then

\begin{verbatim}
		typedef int (1..length(PROT)) PROT_INDEX;
\end{verbatim}

defines a new type {\tt PROT\_INDEX} that indexes a residue in {\tt PROT}. Any
assignment that violates this, e.g.

\begin{verbatim}
		PROT_INDEX x;
		x = length(PROT) + 1;   /* illegal */
\end{verbatim}

will cause a run-time type error (in this case, an extra-clever compiler
with dynamic flow analysis could catch the error at compile-time, but in
general this is impossible).

The range operator {\tt ..} is stolen from Perl. I expect that I won't
implement this type at first, although it's useful for Genewise/HMMER.

\begin{verbatim}
	G parse (S1 [,S2 [,S3...]])
\end{verbatim}

This type represents a parse of sequence {\tt S1} (and possibly sequences {\tt S2} and
{\tt S3}, or more; the exact number of sequenecs is specified by the grammar)
according to grammar {\tt G}.

As with the {\tt int(A..B)} type above, a parameter of this type is {\em defined} to
be within bounds (i.e. a legal parse of the sequences according to the
grammar). This allows us to write simple expressions for well-known
algorithms. e.g. if {\tt SW} is the Smith-Waterman grammar, then the Smith-
Waterman algorithm itself looks like this:

\begin{verbatim}
		typedef SW parse (query, target) qt_alignment;
		qt_alignment optimal_qt_alignment;
		optimal_qt_alignment = argmax (qt_alignment A) { translate (A) };
\end{verbatim}

Here we have used the iterator {\tt argmax} and the built-in function
{\tt translate}. More on these, and on grammars, to follow.


\subsection{Iterators}

\begin{verbatim}
	for     (DISCRETE_TYPE var) { ... statements ... }
	sum     (DISCRETE_TYPE var) { ... expression ... }
	product (DISCRETE_TYPE var) { ... expression ... }
	argmax  (DISCRETE_TYPE var) { ... expression ... }
	sample  (DISCRETE_TYPE var) { ... expression ... }
\end{verbatim}

The model for all of these is {\tt for}. Unlike C, this does not signify a
loop (although it may be executed as such in the compiler), since there is
no flow of control in Telegraph.

Rather, the for construct specifies that the block-enclosed statements are
true for {\em every} value of {\tt var} (note that {\tt var} must have a discrete type for
this to be practical).

Similarly, {\tt sum}, {\tt product} and {\tt argmax} apply the operators {\tt +}, {\tt *} and
{\tt max} to the enclosed expression for all values of {\tt var}. In the case of
{\tt sum} and {\tt product}, the final value has the same type as {\tt expression}; in
the case of {\tt argmax}, it has the same type as {\tt var}, i.e. {\tt DISCRETE\_TYPE}.

{\tt sample} is slightly different, in that it is (sort of) non-deterministic.  
It samples a variable of type {\tt DISCRETE\_TYPE} from the probability
distribution induced by {\tt var}.

\subsection{Grammars}

\begin{verbatim}
	grammar G {
	  terminal T;
	  nonterminal N;
	rule1: NONTERMINALS -> SYMBOLS { PROBABILITY }
		             | SYMBOLS { PROBABILITY }
		  	     | SYMBOLS { PROBABILITY }
			    ... more right-hand-side options ...
			     ;
        rule2: ... more rules ...
	};
\end{verbatim}

This construct describes a grammar. Formal grammars are defined elsewhere
and I won't go into them here. Basically, the {\tt terminal} keyword names the
element types of the sequence(s) that this grammar can parse, whereas the
{\tt nonterminal} keyword identifies names of nonterminals that can be used in
production rules.

The production rules use a similar format to {\tt yacc}/{\tt bison}. The separator for
the left- and right-hand sides of rules is {\tt ->} (in {\tt yacc}, it is {\tt :}, but
Telegraph uses this for labelling rules). {\tt SYMBOLS} is a whitespace-
separated list of symbols (nonterminals or terminals). Where {\tt yacc} has
semantic actions in {\tt \{ ... \}} blocks, Telegraph has probability
expressions.

A sequence {\tt S} may be parseable by this grammar if it has type
{\tt sequence<T>}. A parse {\tt P} of {\tt S} using this grammar has type {\tt G parse (S)}.
The probability of this parse is given by the function {\tt translate(P)}.
Thus the Forward sum-over-alignments for {\tt S} is, notationally

\begin{verbatim}
		forward = sum (G parse (S) P) { translate(P) };
\end{verbatim}

Just because we can write this down in Telegraph doesn't mean it can be
evaluated. In fact, if {\tt G} is an unrestricted grammar then the question ``are
there any values of P with finite probability'' is undecidable (it is the
halting problem!)

It is however very easy to check if {\tt G} is a ``simple'' grammar, e.g. a
regular grammar (an HMM) or a context-free grammar (for RNA). Most
compilers will complain if given any grammar more complex than this.
However, this freedom is (I think) quite a cool way to allow for future
expansion.

The above is a very terse description of the grammar format. The {\tt yacc}
file may shed a bit more light on things.


\subsection{Built-in functions}

\begin{verbatim}
	length(S)
	translate(P)
	diff(E,X)
\end{verbatim}

{\tt length} gives the length of a sequence {\tt S}.

{\tt translate} gives the probability of a parse {\tt P}. (Formally, it replaces a
parse with its semantic translation according to the grammar, then reduces
this sequence to a single expression by a binary operator, which is {\tt *} by
default. These are details, just there to make the theory nice really.)

{\tt diff} differentiates the expression {\tt E} with respect to the parameter {\tt X}.
This is used to express the forward-backward and inside-outside
algorithms.

\bibliographystyle{unsrt}
\bibliography{alignment}

\end{document}
