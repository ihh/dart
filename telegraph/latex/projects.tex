\documentclass{article}
\begin{document}

\title{Telegraph projects}
\maketitle

\section{Setup}

\begin{itemize}
\item Build desk, monitor stand from flatpacks
\item Assemble computer, LCD, mouse, keyboard
\item Burn latst Debian install CD
\item Connect computer to a network point
\item Install Linux; take notes
\item Get X working
\item (Optional) Get wireless card working
\item (Optional) Recompile kernel
\item (Optional) Set up RAID
\item (Optional) Set up USB keyboard drivers and other kernel modules
\end{itemize}

\section{Code to test properties of grammars}

\subsection{Tests for regular grammars}

Write functions to test if the grammar specified by a Telegraph file
is (i) left-regular, (ii) right-regular.

\subsection{Tests that a grammar can terminate}

Write a function to test if a grammar can ``terminate'',
starting from a given nonterminal symbol;
that is, whether there is a path from that nonterminal to ``{\tt end}''.

\section{Code relating to Telegraph expressions}

\subsection{Note on expression binding}

Telegraph expressions can include array indices of the form
``{\tt \$1}'', ``{\tt \$2}'' (etc.), corresponding to the first, second (etc.)
terms on the right-hand-side of a rule expression.

For example, consider the following (incomplete) grammar

\begin{verbatim}
term RNA { A = "a", C = "c", G = "g", T = "t" };
nonterm S;
prob p[RNA][RNA];
S -> RNA S RNA S { p[$1][$3] };
\end{verbatim}

(It's incomplete because it does not terminate.)

To evaluate the final expression, ``{\tt p[\$1][\$3]}'',
we need to {\em bind} the rule so that ``{\tt \$1}'' and ``{\tt \$3}''
are bound to specific {\tt RNA} nucleotides.

To bind a rule, we need to specify subsequence co-ordinates for
every grammar symbol on the right-hand side of the rule; in particular,
the first {\tt RNA} symbol ({\tt \$1}) and the third ({\tt \$3}).

\subsection{Implement simple expression evaluation}

Write a recursive function to implement a bound Telegraph ``simple expression''
(i.e. addition, multiplication, division, probability parameters
and constants).

\subsection{Implement simple expression derivative evaluation}

Write a recursive function to implement the log-derivative of a bound
Telegraph ``simple expression'', with respect to a log-parameter.
That is, for some parameter $x$ and some expression $f(x,\ldots)$,
evaluate the following

\[
\frac{d \left( \log f(x,\ldots) \right)}{d (\log x)}
= \frac{d f(x,\ldots)}{dx} \times \frac{x}{f(x,\ldots)}
\]

\subsection{Unbind an expression}

Given a bound simple expression, write a function to unbind the
expression, i.e. create an equivalent expression
that does not contain any ``{\tt \$}$x$'' indices.

\subsection{Create the likelihood expression for a parse tree}

By multiplying together unbound expressions for the constituent rules,
create the expression that corresponds to the likelihood of
a parse tree.

\subsection{Recursive copy and delete}

Write recursive functions to (i) copy and (ii) delete an expression tree.

\section{Dynamic programming codes}

The following projects involve writing dynamic programming codes
for the ``scripting'' functions of Telegraph.
These implement the various grammar likelihood-related
functions whose (virtual) prototypes are in the file ``{\tt tgpoly.h}''.

In all cases, left-regular grammars (HMMs) can be handled more efficiently than
general stochastic context-free grammars (SCFGs).
Generally, for $N$ sequences of length $L$,
HMMs take $O(L^N)$ time and memory,
whereas SCFGs take $O(L^{2N})$ memory and $O(L^{3N})$ time.
Therefore, it's worth going to the trouble of doing two separate
implementations.

\subsection{The {\tt pmax} function}

Finding the maximum-likelihood parse for a grammar,
given observed sequences.
For HMMs (left-regular grammars), this is the Viterbi algorithm;
for SCFGs, it is the CYK algorithm.

\subsubsection{Viterbi implementation for left-regular grammars}

\subsubsection{CYK implementation for SCFGs}

\subsubsection{Viterbi traceback implementation for left-regular grammars}

\subsubsection{CYK traceback implementation for SCFGs}

\subsection{The {\tt psum} function}

Finding the total likelihood (summed over all parses) for a grammar,
given observed sequences.
For HMMs (left-regular grammars), this is the Forward algorithm;
for SCFGs, it is the Inside algorithm.

\subsubsection{Forward implementation for left-regular grammars}

\subsubsection{Inside implementation for SCFGs}

\subsection{The {\tt psample} function}

Sampling a random parse for a set of observed sequences.

\subsubsection{Sampled Forward traceback implementation for left-regular grammars}

\subsubsection{Sampled Inside traceback implementation for SCFGs}

\subsection{The {\tt dlog} function}

Finding the likelihood derivatives,
which are related to the posterior probabilities
that a particular rule was used to derive a particular set of subsequences.
For HMMs (left-regular grammars), this uses the Forward-Backward algorithm;
for SCFGs, it uses the Inside-Outside algorithm.

\subsubsection{Outside implementation for SCFGs}

\subsubsection{Backward implementation for left-regular grammars}

\subsubsection{Derivative implementation for left-regular grammars}

\subsubsection{Derivative implementation for SCFGs}

\section{Telegraph-related utility codes}

\subsection{Convert a Telegraph parse tree to a Stockholm alignment}

Write a function that outputs a Telegraph parse tree as a Stockholm alignment
(see e.g. {\tt biowiki.org/StockholmFormat}).

\subsection{Telegraph as a distributed computing platform}

\subsubsection{Telegraph XML generation using {\tt yaxx}}

Write a version of the Telegraph compiler that generates XML,
using the {\tt yaxx} program at {\tt yaxx.sourceforge.net}.

\subsubsection{Web interface to Telegraph}

Build a simple web UI to Telegraph, running on a dedicated Linux server.

\subsubsection{Use Telegraph for remote job scheduling}

Build a system that dispatches Telegraph grammars/scripts (``telegrams'')
over a network connection to a compute farm, and collates the results.

\subsubsection{Improve the scripting part of the Telegraph language}

Telegraph's scripting capabilities could probably be more elegant.
For example, by using concepts from Matlab, Maple, Mathematica, etc,
to clarify the idea of ``simplifying'' an expression in various ways
(from a {\tt psum}, possibly via an explicit polynomial, to a constant value).
Comment, with proposals for improvements.

\subsection{MCMC multiple alignment}

These projects use Telegraph to implement some of the ideas behind
Evolutionary HMMs and Evolutionary SCFGs.

\subsubsection{Utilities to decompose and re-compose multiple alignments}

Using the DART libraries, or otherwise,
write small utilities to do the following:

\begin{itemize}
\item Given, as input, a Stockholm multiple alignment
including a phylogenetic ``guide tree''
 (containing all the sequence names in the alignment),
break the alignment down into a Stockholm database of pairwise alignments,
and print this on the output.
\item Given, as input, a Stockholm multiple alignment
and a series of subtrees (each of which contains some
subset of the sequence names in the alignment),
break the alignment down into a Stockholm database of smaller alignments
(one for each of the phylogenetic trees),
and print this on the output.
\item Given, as input, a Stockholm database of pairwise alignments,
each corresponding to a single branch of an assumed ``guide tree'',
print on the output a single Stockholm multiple alignment
(with the guide tree).
\end{itemize}

Feel free to find a more elegant way to fulfil these tasks,
e.g. by combining the first two tasks into a single tool.

\subsubsection{Implement a Handel-style alignment sampler using Telegraph}

Use a scripting language as well as Telegraph.

See Holmes and Bruno, Bioinformatics, 2001.

\subsubsection{Extend the Handel-style alignment sampler to sample trees too}

This involves adding some extra MCMC ``moves'' to the alignment sampler.

\subsection{Consensus alignment}

(Collaboration with Sjolander group.)

Implement a tool to make a ``consensus'' alignment from a
Stockholm database of multiple alignments.


\section{Applications of Telegraph and DART}

\subsection{Using stemloc (and/or evoldoer) to look for
RNA regulatory structures in HIV and other viruses}

(UndergraduateClass exercise.)

Which of the following RNA structural elements can be detected
by {\tt stemloc} or {\tt evoldoer},
(i) if a suitably ``trimmed'' set of sequences (or alignment)
is presented to the program;
(ii) if the whole virus genome is tested?
Try playing with the program settings to find the ones that work best.
(Note: (ii) probably won't work very well due to the large size of
the sequences involved, but try it anyway to see what happens.)

\begin{itemize}
\item The HIV Rev Response Element (RRE);
\item The SARS s2m RNA;
\item The elements identified by Hofacker {\em et al}
(see e.g. Hofacker, Stadler and Stocsits, Bioinformatics 2004).
\end{itemize}

\subsection{Curation of viral regulatory RNAs}

By searching the literature for conserved RNA secondary structures
implicated in viral function (packaging, regulation of viral gene expression,
regulation of host gene expression...),
make a curated dataset of alignments of such elements,
in Stockholm format.

The elements should not be in RFAM yet, but should be
in a format suitable for submission to RFAM.


\subsection{Identify fruitfly microRNAs}

(UndergraduateClass exercise.)

Develop an exercise whereby the student is presented with several
alignments, one of which contains a microRNA, one contains a tRNA,
and several others are duds (or contain protein-coding genes or
other conserved sequences that aren't ncRNA genes).

The student should use pfold, xfold or QRNA.

Optional: make the exercise more challenging by using longer alignments
and the {\tt pfoldscan.pl} script.

\subsection{Benchmarking stemloc, tkfalign+xfold,
and other multiple alignment tools for RNA}

Use the {\tt sankoff-paper/benchmark/benchmark.pl} script in
the {\tt sankoff-paper} CVS repository on buffy as a starting point.

\section{Documentation-related projects}

\subsection{Documentation of the Linux computer setup process}

Make it easy for others to do what you have done (setting up the
Linux computer). Write a wiki page describing the process.

\subsection{Latex document describing the Telegraph language}

Write a succinct, introductory Latex document describing the Telegraph
language, including a summary of the grammar (from the {\tt gram.y} file).

\subsection{Header file comments}

Add a short comment paragraph to the top of every ``{\tt .h}'' file,
outlining the various data structures and functions used by the
Telegraph compiler/interpreter.

\subsection{Telegraph user's guide}

Write a page or two describing Telegraph to a novice user.
The text should be biologist-friendly.

\subsection{Wiki pages}

Update the wiki pages for Telegraph.

\end{document}
