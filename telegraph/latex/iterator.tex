\documentclass{article}
\begin{document}

\title{Telegraph dynamic programming}
\maketitle

\section{Integer vector equations}

Dynamic programming in Telegraph involves iterating
in a well-defined order over the
solutions to equations on integer vector spaces.

Let ${\bf y}, {\bf x}^{(1)}, {\bf x}^{(2)}, \ldots {\bf x}^{(m)}$
be $N$-dimensional vectors of non-negative integers.

The ${\bf x}^{(k)}$ are variables while ${\bf y}$ is a constant.

Consider the solutions to the following vector equation

\begin{equation}
\sum_{k=1}^m {\bf x}^{(k)} = {\bf y}
\label{eq:intsum}
\end{equation}

A one-dimensional example, with $N=1$, $y=3$ and $m=3$, is shown in
Table~\ref{tab:y3m3}.
A two-dimensional example, with $N=2$, ${\bf y}=(3,2)$ and $m=3$,
is shown in Tables~\ref{tab:y32m3part1} and~\ref{tab:y32m3part2}.

Note how the table rows are sorted.
Ignoring the elements of ${\bf x}^{(m)}$,
the remaining undetermined integers are incremented in the following order:

\[
x^{(1)}_1, x^{(2)}_1, \ldots x^{(m-1)}_1,
\ x^{(1)}_2, x^{(2)}_2, \ldots x^{(m-1)}_2,
\ \ldots\ \ldots
\ x^{(1)}_N, x^{(2)}_N, \ldots x^{(m-1)}_N
\]

That is, $x^{(k)}_n$ is incremented in order of increasing $k$,
then increasing $n$.

Let the index $I\left({\bf X}\right)$
be the ranking of a solution ${\bf X} = ({\bf x}^{(1)} \ldots {\bf x}^{(m)})$
in such an ordered iteration, starting at zero,
as shown in Tables~\ref{tab:y3m3} through~\ref{tab:y32m3part2}.

Let $M({\bf y})$ denote the total number of solutions for a particular value of $m$.

The following two subroutines must be implemented for
dynamic programming to SCFGs:

\begin{itemize}
\item Iterate over the solutions (in order) for any $m$, $N$ and ${\bf y}$. \\
That is, given some subroutine $S$, call $S({\bf X})$ for all solutions ${\bf X}$. \\
Call this subroutine {\tt Iterate}$({\bf y},m,S)$.
\item Implement $I$ for the special case $m=3$. \\
Call this subroutine {\tt Index}$({\bf y},{\bf X})$.
\end{itemize}

\section{Indexing subsequences}

The implementation of the {\tt Index} function can make use of the following closed form for $I$ when $m=3$.

Consider first the one-dimensional case, $N=1$.
Then
$I\left({\bf X}\right) \equiv I'(x^{(1)}_1,x^{(2)}_1;y_1)$,
where

\begin{eqnarray*}
I'(u,v;w) & = & u + \sum_{k=0}^{v-1} (w + 1 - k) \\
& = & u + \frac{1}{2}(v+1)(2w+4-v) - (w+2)
\end{eqnarray*}

The total number of solutions is $M'(y_1) = \frac{1}{2}(y_1+1)(y_1+2)$.

In the more general multidimensional case, where $m=3$ (still) but $N \geq 1$,

\[
I\left({\bf X}\right)
= \sum_{n=1}^N I'(x^{(1)}_n,x^{(2)}_n;y_n) \prod_{j=1}^{n-1} M'(y_j)
\]

The total number of solutions is

\begin{eqnarray*}
M({\bf y}) & = & \prod_{n=1}^N M'(y_n) \\
& = & \frac{1}{2^N} \prod_{n=1}^N (y_n+1) (y_n+2)
\end{eqnarray*}

\section{Iterating over subsequences}

The main DP loop iterates over subsequences, inside to outside.

Let ${\bf L}$, ${\bf s}$, ${\bf a}$ and ${\bf e}$ be $N$-dimensional vectors of nonnegative integers.

There are $N$ terminal sequences and ${\bf L}$ is the vector of sequence lengths.

We iterate over a multi-subsequence whose co-ordinates are given by ${\bf s}$, ${\bf a}$ and ${\bf e}$,
which satisfy the constraint

\[
{\bf s} + {\bf a} + {\bf e} = {\bf L}
\]

The $n$'th terminal subsequence has length $a_n$
and is flanked by $s_n$ symbols on the left and $e_n$ on the right.

We require that ``inside'' subsequences ${\bf E}=({\bf s},{\bf a},{\bf e})$
are visited before ``outside'' subsequences ${\bf E}'=({\bf s}',{\bf a}',{\bf e}')$.
This means that if, for all $n$, $s_n \geq s'_n$ and $e_n \leq e'_n$,
then ${\bf E}$ is visited before ${\bf E}'$.

This is satisfied by our ordered iteration over solutions of~(\ref{eq:intsum}),
using the following mapping:

\begin{eqnarray*}
      {\bf y} & = & {\bf L} \\
            m & = & 3 \\
      {\bf X} & = & {\bf E}
\end{eqnarray*}

i.e.
 ${\bf x}^{(1)} = {\bf s}$,
 ${\bf x}^{(2)} = {\bf a}$ and
 ${\bf x}^{(3)} = {\bf e}$.

The total storage required for each DP matrix is
$W \times M({\bf L}) \times Z$,
where $W$ is the number of nonterminal symbols in the grammar
and $Z$ is the storage size of a probability (e.g. 4 bytes).


\section{Matching rules}
\label{sec:MatchRule}

Suppose we have a grammar rule with $R$ symbols on the right-hand side

\[
A\ \to\ B_1\ B_2\ \ldots\ B_R
\]

Here $A$ is a nonterminal and the $B_r$ may be terminals or nonterminals.

Let ${\bf a}, {\bf b}^{(1)} \ldots {\bf b}^{(R)}$ be $N$-dimensional vectors of non-negative integers.

The vector ${\bf a}$ represents the amount of sequence emitted by the left-hand symbol $A$.
The ${\bf b}^{(r)}$ represent the amount of sequence emitted by each right-hand symbol $B_r$.
Those ${\bf b}^{(r)}$ corresponding to terminal symbols are fixed,
while those corresponding to nonterminals are variable.

If $B_r$ is a terminal symbol in sequence $n$,
then let $b^{(r)}_j=1$ if $j=n$ and $0$ if $j \neq n$.
If $B_r$ is a nonterminal, then ${\bf b}^{(r)}$ is undetermined.

The equation to be solved is

\begin{equation}
\sum_{r=1}^R {\bf b}^{(r)} = {\bf a}
\label{eq:rule}
\end{equation}

Equation~(\ref{eq:rule}) reduces to~(\ref{eq:intsum}), as follows.

Define the indicator function $T(r)$
to be $1$ if $B_r$ is a terminal, and $0$ if $B_r$ is a nonterminal.

Let $K(r)= \sum_{s=1}^r (1-T(r))$ be the number of nonterminals
in the first $r$ symbols on the right-hand side, i.e. in $B_1 \ldots B_r$.

Then~(\ref{eq:rule}) can be rewritten as

\[
\sum_{r=1}^R (1-T(r)) {\bf b}^{(r)} = {\bf a} - \sum_{r=1}^R T(r) {\bf b}^{(r)}
\]

which reduces to~(\ref{eq:intsum}) using the following mapping

\begin{eqnarray*}
         {\bf y} & = & {\bf a} - \sum_{r=1}^R T(r) {\bf b}^{(r)} \\
               m & = & M_R \\
{\bf x}^{(K(r))} & = & {\bf b}^{(r)}
 \quad \quad \quad \quad \quad \quad \forall\ r:\ T(r)=0
\end{eqnarray*}


\section{Matching characters}

Two kinds of terminal symbols may appear on the right-hand side of rules:
(i) exact character matches and (ii) wildcards.

When attempting to match a rule at a given set of co-ordinates,
the rule should be rejected (i.e. found to have probability zero)
if any of the exact character matches do not match the sequence data.

The matching character values of any wildcards should, at around this time,
be copied into a suitable array, so that the values of the Telegraph
wildcard matches ({\tt \$1}, {\tt \$2} etc.) can be looked up
during evaluation of the rule probability expression.

Suppose symbol $B_r$ is a wildcard in terminal alphabet $n$.
Let $\alpha$ be the index of the matching character in the $n$'th
terminal sequence, starting from $1$.
Then, for a given set of co-ordinates ${\bf b}^{(1)} \ldots {\bf b}^{(R)}$
satisfying~(\ref{eq:rule}), the index is

\[
\alpha = a_n + \sum_{k=1}^r b_n
\]


\newpage

\begin{table}
\begin{center}
\begin{tabular}{r|lll}
$I$ & $x^{(1)}$ & $x^{(2)}$ & $x^{(3)}$
\\ \hline
$0$ & $0$ & $0$ & $3$ \\
$1$ & $1$ & $0$ & $2$ \\
$2$ & $2$ & $0$ & $1$ \\
$3$ & $3$ & $0$ & $0$ \\
$4$ & $0$ & $1$ & $2$ \\
$5$ & $1$ & $1$ & $1$ \\
$6$ & $2$ & $1$ & $0$ \\
$7$ & $0$ & $2$ & $1$ \\
$8$ & $1$ & $2$ & $0$ \\
$9$ & $0$ & $3$ & $0$
\end{tabular}
\end{center}
\caption{
\label{tab:y3m3}
Solutions to~(\ref{eq:intsum}) with $N=1,y=3,m=3$.
}
\end{table}

\newpage
\begin{twocolumn}

\begin{table}
\begin{center}
\begin{tabular}{r|lll}
$I$ & ${\bf x}^{(1)}$ & ${\bf x}^{(2)}$ & ${\bf x}^{(3)}$
\\ \hline
$0$ & $(0,0)$ & $(0,0)$ & $(3,2)$ \\
$1$ & $(1,0)$ & $(0,0)$ & $(2,2)$ \\
$2$ & $(2,0)$ & $(0,0)$ & $(1,2)$ \\
$3$ & $(3,0)$ & $(0,0)$ & $(0,2)$ \\
$4$ & $(0,0)$ & $(1,0)$ & $(2,2)$ \\
$5$ & $(1,0)$ & $(1,0)$ & $(1,2)$ \\
$6$ & $(2,0)$ & $(1,0)$ & $(0,2)$ \\
$7$ & $(0,0)$ & $(2,0)$ & $(1,2)$ \\
$8$ & $(1,0)$ & $(2,0)$ & $(0,2)$ \\
$9$ & $(0,0)$ & $(3,0)$ & $(0,2)$
\\
$10$ & $(0,1)$ & $(0,0)$ & $(3,1)$ \\
$11$ & $(1,1)$ & $(0,0)$ & $(2,1)$ \\
$12$ & $(2,1)$ & $(0,0)$ & $(1,1)$ \\
$13$ & $(3,1)$ & $(0,0)$ & $(0,1)$ \\
$14$ & $(0,1)$ & $(1,0)$ & $(2,1)$ \\
$15$ & $(1,1)$ & $(1,0)$ & $(1,1)$ \\
$16$ & $(2,1)$ & $(1,0)$ & $(0,1)$ \\
$17$ & $(0,1)$ & $(2,0)$ & $(1,1)$ \\
$18$ & $(1,1)$ & $(2,0)$ & $(0,1)$ \\
$19$ & $(0,1)$ & $(3,0)$ & $(0,1)$
\\
$20$ & $(0,2)$ & $(0,0)$ & $(3,0)$ \\
$21$ & $(1,2)$ & $(0,0)$ & $(2,0)$ \\
$22$ & $(2,2)$ & $(0,0)$ & $(1,0)$ \\
$23$ & $(3,2)$ & $(0,0)$ & $(0,0)$ \\
$24$ & $(0,2)$ & $(1,0)$ & $(2,0)$ \\
$25$ & $(1,2)$ & $(1,0)$ & $(1,0)$ \\
$26$ & $(2,2)$ & $(1,0)$ & $(0,0)$ \\
$27$ & $(0,2)$ & $(2,0)$ & $(1,0)$ \\
$28$ & $(1,2)$ & $(2,0)$ & $(0,0)$ \\
$29$ & $(0,2)$ & $(3,0)$ & $(0,0)$
\end{tabular}
\end{center}
\caption{
\label{tab:y32m3part1}
Solutions to~(\ref{eq:intsum}) with $N=2, {\bf y}=(3,2), m=3$
(cont. in Table~\ref{tab:y32m3part2}).
}
\end{table}

\begin{table}
\begin{center}
\begin{tabular}{r|lll}
$I$ & ${\bf x}^{(1)}$ & ${\bf x}^{(2)}$ & ${\bf x}^{(3)}$
\\ \hline
$30$ & $(0,0)$ & $(0,1)$ & $(3,1)$ \\
$31$ & $(1,0)$ & $(0,1)$ & $(2,1)$ \\
$32$ & $(2,0)$ & $(0,1)$ & $(1,1)$ \\
$33$ & $(3,0)$ & $(0,1)$ & $(0,1)$ \\
$34$ & $(0,0)$ & $(1,1)$ & $(2,1)$ \\
$35$ & $(1,0)$ & $(1,1)$ & $(1,1)$ \\
$36$ & $(2,0)$ & $(1,1)$ & $(0,1)$ \\
$37$ & $(0,0)$ & $(2,1)$ & $(1,1)$ \\
$38$ & $(1,0)$ & $(2,1)$ & $(0,1)$ \\
$39$ & $(0,0)$ & $(3,1)$ & $(0,1)$
\\
$40$ & $(0,1)$ & $(0,1)$ & $(3,0)$ \\
$41$ & $(1,1)$ & $(0,1)$ & $(2,0)$ \\
$42$ & $(2,1)$ & $(0,1)$ & $(1,0)$ \\
$43$ & $(3,1)$ & $(0,1)$ & $(0,0)$ \\
$44$ & $(0,1)$ & $(1,1)$ & $(2,0)$ \\
$45$ & $(1,1)$ & $(1,1)$ & $(1,0)$ \\
$46$ & $(2,1)$ & $(1,1)$ & $(0,0)$ \\
$47$ & $(0,1)$ & $(2,1)$ & $(1,0)$ \\
$48$ & $(1,1)$ & $(2,1)$ & $(0,0)$ \\
$49$ & $(0,1)$ & $(3,1)$ & $(0,0)$
\\
$50$ & $(0,0)$ & $(0,2)$ & $(3,0)$ \\
$51$ & $(1,0)$ & $(0,2)$ & $(2,0)$ \\
$52$ & $(2,0)$ & $(0,2)$ & $(1,0)$ \\
$53$ & $(3,0)$ & $(0,2)$ & $(0,0)$ \\
$54$ & $(0,0)$ & $(1,2)$ & $(2,0)$ \\
$55$ & $(1,0)$ & $(1,2)$ & $(1,0)$ \\
$56$ & $(2,0)$ & $(1,2)$ & $(0,0)$ \\
$57$ & $(0,0)$ & $(2,2)$ & $(1,0)$ \\
$58$ & $(1,0)$ & $(2,2)$ & $(0,0)$ \\
$59$ & $(0,0)$ & $(3,2)$ & $(0,0)$
\end{tabular}
\end{center}
\caption{
\label{tab:y32m3part2}
Solutions to~(\ref{eq:intsum}) with $N=2, {\bf y}=(3,2), m=3$
(begun in Table~\ref{tab:y32m3part1}).
}
\end{table}

\end{twocolumn}

\newpage

\end{document}
